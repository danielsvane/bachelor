% !TEX root = ../main.tex

The RNA strands for the translator couldn't be transcribed, possible due to secondary structures. It is not possible to avoid the secondary structures due to the restrictions of the translator design, so the RNA will have to be synthesized to create translators in RNA. The experiment was almost identical to the one in Picuri et al 2009, and there is nothing to suggest that the strand displacement shouldn't have worked using RNA instead of DNA \cite{Picuri2009}. The authors even use RNA as the input for the DNA translators.

The simulations of the neural network did show some promising results. It was possible to create multiple perceptrons of varying input size, and train them to different truth tables (\cref{2_and,2_or,3_and,3_1_or,3_2_or}). The system requires no input from the user after having defined the desired truth table, and should theoretically be able to realize any truth tables of any size as long as they don't require negative weights, and are linearly separable.

The perceptron in this experiment is still only a stripped down version of the networks in Qian et all 2011, as it was not possible to apply the dual-rail logic needed for avoiding negative weights. Furthermore, the perceptrons trained in this experiment, could have been made much simpler by the logic gate system the same authors designed recently \cite{Thubagere2017}. The only novel thing that came out of this experiment was the simplified training algorithm for the perceptrons without dual-rail logic, and the automatic generation of the perceptron based on a truth table.

There is also the problem of applying the simulations to in vitro reactions. Qian et all 2011 found some general guidelines for adjusting the concentrations that seemed to work for some networks, but it can't be expected that the sequences and concentrations found in the simulation will always work in vitro. They also express concern for moving to in vivo.

Still, the preliminary work done in this experiment could in theory be used to create a method for designing RNA/DNA detection kits. With some more work, the training algorithm and input translation could be combined, and tested in vitro.
