% !TEX root = ../main.tex
Logic circuits using DNA and RNA has many interesting applications in diagnostics and treatment. An example is cancer detection, where miRNA's can be used as biomarkers \cite{Peng2016}. These biomarkers can be used as inputs for logic circuits, which can be designed to activate fluorescence signals \cite{Seelig2006} or enzymes \cite{Engelen2016} when combinations of biomarkers are present or absent. For example, a circuit could be designed to activate when 2 unique miRNA's are present at the same time, or when a certain protein is present.

A different approach to building logic circuits is using neural networks. One of the advantages of this approach, is that large-scale networks does not have to be designed by someone with knowledge of logic gates and circuitry. The inputs of the circuit simply have to be defined along with the required output, and the neural network can be trained to give the desired functionality.

The neural network approach to strand displacement circuits has already been developed \cite{Qian2011}. The design is based on the seesaw gate motif, which requires that the inputs to the network must be have very specific sequences. To use custom input sequences, like the miRNAs from cancer cells, the sequences have to be translated. This has also already been done, using two linked strand displacement reactions \cite{Picuri2009}.

This project is split into two parts.

The first part aims to lay out the groundwork for a software where the miRNAs one wishes to detect can be entered. The strands and concentrations required for a neural network that implements a desired truth table is then calculated, and could be readily ordered and used as a biosensor.

The second part aims to translate one RNA sequence into another RNA sequence, and measuring the outcome by a fluorescent reporter. This is almost identical to the experiment in \cite{Picuri2009}, but using RNA instead of DNA in the translation reactions. This is mostly done to show that the translators could also use RNA, but also to get some experience in a laboratory.
