% !TEX root = ../main.tex
\section{Introduction}
Logic circuits using DNA and RNA has many interesting applications in diagnostics and treatment. An example is cancer detection, where miRNA's can be used as biomarkers \cite{Peng2016}. These biomarkers can be used as inputs for logic circuits, which can be designed to activate FRET signals \cite{Seelig2006} or enzymes \cite{Engelen2016} when combinations of biomarkers are present or absent. For example, a circuit could be designed to activate when 2 unique miRNA's are present at the same time, or when a certain protein is present. This is represented schematically in figure \ref{example_circuit}.

\begin{figure}[h!]
\centering
\begin{circuitikz} \draw
  (0,3) node[and port] (and1) {}
  (2,2) node[or port] (or1) {}
  (and1.in 1) node [anchor=east] {miRNA 1}
  (and1.in 2) node [anchor=east] {miRNA 2}
  (and1.out) -| (or1.in 1)
  (or1.in 2) node [anchor=east] {Protein 1}
  (or1.out) node [anchor=west] {FRET or enzymatic activity}
;\end{circuitikz}
\label{example_circuit}
\caption{Example of logic circuit}
\end{figure}

The logic gates can be made using strand displacement \cite{Zhang2011}, where the outputs of one gate can be linked to the input of another by unique DNA or RNA sequences. The strand displacement technique can for example be used to create an AND gate, as seen in figure \ref{example_and}.

\begin{figure}[h!]
\centering


\begin{scaletikzpicturetowidth}{\columnwidth}
\begin{tikzpicture}[scale=\tikzscale]


\def\strandtwo{
  \draw[<-](4,0.3) -- node[above] {d} (5.9,0.3);
  \node[label=above:{e}] at (6,1) {};
  \draw plot [smooth, tension=1] coordinates {(5.9, 0.3) (5.9, 0.4) (5.7, 0.7) (6, 1) (6.3, 0.7) (6.1, 0.4) (6.1, 0.3)};
  \draw(6.1,0.3) -- node[above] {f} (8,0.3);
}

\def\strandone(#1,#2){
  \begin{scope}[shift={(#1,#2)}]
    \draw[<-](0,0.3) -- node[above] {a} (1,0.3);
    \draw(1,0.3) -- node[above] {b} (3,0.3);
    \draw(3,0.3) -- node[above] {c} (4,0.3);
  \end{scope}
}

\def\strandthree{
  \draw(1,0) -- node[below] {b*} (3,0);
  \draw(3,0) -- node[below] {c*} (4,0);
  \draw(4,0) -- node[below] {d*} (6,0);
  \draw[->](6,0) -- node[below] {f*} (8,0);
}

\def\inputstrand(#1,#2){
  \begin{scope}[shift={(#1,#2)}]
    \draw[](0,0.3) -- node[below] {a*} (1,0.3);
    \draw(1,0.3) -- node[below] {b*} (3,0.3);
    \draw[->](3,0.3) -- node[below] {c*} (4,0.3);
  \end{scope}
}

\strandone(0,0)
\strandtwo
\strandthree

\node at (9,0.2) {+};

\inputstrand(10,-0.1)

\draw[->](7, -1) -- (7, -3);
\draw[->](7, -1) to [out=-90, in=0] (6, -2);
\draw[->](8, -2) to [out=180, in=90] (7, -3);

\strandone(0,-2)
\inputstrand(0,-2.3)

\end{tikzpicture}
\end{scaletikzpicturetowidth}

\label{example_and}
\caption{Example of logic circuit}
\end{figure}
